%!TEX root = ../memoria.tex


\vspace{20mm}
This work investigates the inherent risks in the creation and use of applications based on Large Language Models (LLM) in the industry. Its focus is on Natural Language Processing (NLP), specifically text generation, excluding other forms of generative artificial intelligence. The project is based on the experience of developing an application using LLM and analyzes the risks that can affect both the development team and the obtained results. The risk factors in the creation and use of these applications are determined, using as a case study a jurisprudence search project in environmental courts. The methodology employed includes the creation of the project, a practical example of use, and the evaluation of risks at each stage of the process. The goal is to provide a structure for applications that use LLM, analyze the problems and risks associated with the use of information to feed these models, including a complete ETL (Extract, Transform, Load) process.

\paragraph{Keywords.}
Large Language Models (LLM),
Text Generation,
Risk,
Artificial Intelligence,
ETL Process,
Environmental Jurisprudence

\vspace{20mm}


%\begin{framed}
%\noindent\textbf{Instrucciones para la Plantilla.}
%
%Editar el archivo \inlinecode{/includes/abstract.tex} para modificar los contenidos de esta sección.%
%
%Si no desea incluir un abstract, editar el archivo \inlinecode{/memoria.tex}, y comentar o borrar la sección que se muestra a continuación.
%
%\begin{Verbatim}[frame=lines, label=\inlinecode{/memoria.tex} (extracto)
%				, fontsize=\footnotesize
%				, baselinestretch=1
%				, formatcom=\color{gray}]
%\section*{ABSTRACT}
%\insertFile[plain]{abstract}			% Archivo abstract.tex
%\end{Verbatim}
%
%\end{framed}
