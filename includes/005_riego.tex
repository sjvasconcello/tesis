
\chapter{Evaluación de Riesgos}

\section{Creación del Proyecto}

% Elección de equipos componentes

% Sesgo (Raciales)

% No realizar un analisis previo

% Licencias LLM

% Datos Actualizados

% Costos

% Lesgislacion de datos

% robots.txt

% Pureza de los datos

% Tratamiento de los datos

% Embbeding

% Chunks

% Medatada

% Elección de la base de datos


\section{Uso de la Aplicación}

\subsection{Entrega de contexto adecuado}

Los LLM a menudo presentan alucinaciones, por lo que es esencial reducir la frecuencia de este fenómeno. 
Para lograrlo, la provisión de contexto dentro del prompt no es simplemente precisa, sino que resulta 
absolutamente indispensable. De hecho, la entrega de contexto adecuado dentro del prompt ha demostrado 
ser una medida altamente efectiva para reducir las alucinaciones, logrando una disminución de hasta un 
99.88 porciento. \cite{riego1}

Por consiguiente, la correcta entrega de contexto dentro del prompt desempeña un papel fundamental en 
la generación de respuestas precisas a las consultas. Esto se debe a que, ya sea que el contexto 
proporcionado sea correcto, incorrecto o incluso irrelevante, el modelo de lenguaje lo utilizará como base 
para generar sus respuestas.

En el contexto del proyecto, la generación de respuestas se basa por completo en la entrega de contexto 
dentro del prompt, lo que a veces puede dar lugar a la transmisión de más información de la necesaria 
debido al funcionamiento del framework de Langchain. Esto puede llevar a situaciones en las que el modelo, 
influenciado por la información incorrecta o adicional proporcionada, genere respuestas que no reflejan 
un output con una respuesta en su totalidad correcta.

\subsection{Volatilidad del Mercado}

Hasta la fecha actual, el 06 de noviembre de 2023, la creación de Chatbots utilizando el método RAG se perfilaba como 
una de las tendencias más destacadas en el mercado, siendo posiblemente una de las aplicaciones más prometedoras de los LLM. 
No obstante, en este mismo día, durante la OpenAI DevDay Keynote, se anunciaron novedades significativas, como la entrada en 
escena de los GPTs, que permite personalizar versiones de ChatGPT con instrucciones, conocimiento extra y cualquier otra combinación 
de habilidades \cite{openai2}. Además, se introdujeron otros modelos como gpt-4 turbo, un playground de desarrollo para la herramienta, 
text-to-speech (TTS), entre otros \cite{openai3}.

En este contexto, comprometerse con cualquier tecnología conlleva riesgos, especialmente en este período caracterizado por una 
volatilidad extrema y una inversión extremadamente agresiva en inteligencia artificial. La Inteligencia Artificial Generativa 
continúa evolucionando de manera acelerada, lo que la hace cada vez más disruptiva y más eficiente. Por lo tanto, la investigación 
y la implementación de soluciones de inteligencia artificial centradas en asistentes o chatbots representan un compromiso de alto 
riesgo en este entorno en constante cambio.


% Bloqueos de respuesta

% Proximidad de los vectores

% Alusionaciones

% Conocimiento del funcionamiento de herramientas
