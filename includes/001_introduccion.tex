
\chapter{Introducción}
La inteligencia artificial, también conocida como IA, ha experimentado un notable auge en la industria en los últimos tiempos
de la mano de la llamada Industria 4.0 \cite{intro1}, 
especialmente en el ámbito en areas algo reacias como la administración y finanzas\cite{intro2}. 
Este incremento no se debe necesariamente a un aumento en la capacidad de cómputo, 
ya que esta ha ido creciendo gradualmente a lo largo del tiempo (buscar respaldo). Anteriormente, aunque importante, no generaba tanto 
interés como en la actualidad. No fue sino hasta que OpenAI lanzó ChatGPT el 30 de noviemnbre de 2022 que el público en general 
pudo experimentar, 
probar y comprender de manera más completa la gran revolución llamada inteligencia artificial generativa \cite{intro3}.

De acuerdo con Google, "La inteligencia artificial generativa se refiere al uso de la IA para crear contenido, como texto, imágenes, 
música, audio y videos"\cite{google1}. Gracias a su interfaz amigable, resultó sencillo para personas de diversas industrias descubrir que existía 
una herramienta capaz de generar texto y responder preguntas de manera comprensible, incluso para aquellos que no eran expertos en 
tecnología.

La génesis de esta tesis se basa en la experiencia de llevar a cabo un proyecto utilizando estas tecnologías 
y los riesgos asociados a ellas. En este contexto, entendemos el riesgo como cualquier aspecto que pueda afectar 
tanto al equipo involucrado en la creación del proyecto como a los resultados obtenidos. El proyecto se centró 
en el uso de un modelo de lenguaje de gran envergadura, conocido como LLM por sus siglas en inglés, limitándose a la 
generación de texto. Por lo tanto, no profundizaremos en otros tipos de inteligencia artificial generativa, como la 
generación de imágenes o audio. El enfoque principal de este trabajo se concentra solo
en el area del procesamiento del lenguaje natural aplicados a LLM.


\section{Obejetivos}
% Objetivos
\subsection{Objetivo General}
Determinar los factores de riesgo que pueden llegar a influir, tanto de la creación como del uso de aplicaciones que utilicen modelos grandes de lenguaje (LLM) aplicados a la industria, usando de base el proyecto de búsqueda de jurisprudencia de los tribunales ambientales. 
\subsection{Objetivo Específico}

\begin{enumerate}
\item Determinar una posible estructura de una aplicación usando LLM
\item Desarrollar los estados del Arte del uso de LLM y de los modelos generativo en si
\item Desarrollar los problemas que conlleva el uso de información para alimentar dichos problemas
\item Analizar un proceso de ETL de principio a fin para observar sus posibles riesgos
\end{enumerate}


\section{Metodologia}

La metodología empleada en esta tesis se estructura en torno a tres componentes esenciales: 
la creación del proyecto, un ejemplo de uso concreto y la evaluación de los riesgos asociados 
a cada etapa del proceso, tanto en la fase de desarrollo del proyecto como en su aplicación práctica.
\subsection{Creación del Proyecto}

\noindent Esta fase inicial comprende el desarrollo del proyecto basado en IA generativa. Incluye los siguientes pasos:

\begin{itemize}
    \item \textbf{Definición de Objetivos y Alcance:} Establecimiento claro de los propósitos y límites del proyecto, identificando las metas a alcanzar.
    \item \textbf{Selección de Tecnologías y Herramientas:} Evaluación y elección de las tecnologías y herramientas apropiadas para la implementación del proyecto.
    \item \textbf{Diseño de la Arquitectura:} Desarrollo de la estructura y componentes del proyecto, considerando aspectos de escalabilidad y rendimiento.
    \item \textbf{Implementación y Desarrollo:} Construcción efectiva del proyecto, incluyendo la programación y configuración de la inteligencia artificial generativa.
\end{itemize}
\subsection{Ejemplo de Uso del Proyecto}

Esta fase implica la aplicación práctica del proyecto en un contexto específico, demostrando su funcionalidad y utilidad. 
En este caso nuestro interés mas que en el output que genere la aplicación, es como funciona internamente el proceso cosa 
que para la siguiente etapa sea más fácil 

\subsection{Evaluación de Riesgos}

Se trabajará la identificación y análisis de los riesgos potenciales en cada etapa del proceso, así como los riesgos derivados
del caso de uso. Incluye:

\begin{itemize}
    \item \textbf{Riesgos en la Creación del Proyecto:} Identificación de posibles obstáculos y contratiempos durante la etapa de concepción y desarrollo.
    \item \textbf{Riesgos en el Ejemplo de Uso:} Consideración de los riesgos asociados a la implementación práctica del proyecto en el contexto definido.
\end{itemize}

Esta metodología proporciona un enfoque integral para la creación y aplicación de un proyecto utilizando LLM, permitiendo una evaluación de los riesgos en cada etapa del proceso y en el caso de uso específico. Esto facilita la toma de decisiones informadas y la formulación de estrategias para mitigar posibles contratiempos.



% Contribuciones

% Estructura de la tesis


% Metodologia


% Alcance

% Estado del Arte

% Desarrollo
