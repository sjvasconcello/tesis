
\chapter{Introducción}
La inteligencia artificial, también conocida como IA, ha experimentado un notable auge en la industria en los últimos tiempos
de la mano de la llamada Industria 4.0 \cite{intro1} y dentro del imaginario colectivo gracias a sus aplicacióny y facíl acceso, 
especialmente en ámbitos que durante mucho tiempo fueron algo reacios al cambio, como la administración y las finanzas \cite{intro2}. 
Este incremento no se debe necesariamente a un aumento en la capacidad de cómputo a la que tenemos acceso, 
ya que esta capacidad ha ido creciendo gradualmente a lo largo del tiempo. Dicho lo anterior, aunque muy importante e investigado, 
la inteligencia artificial no generaba tanto interés como en la actualidad, porque esta estaba reservada para investigadores e implamentaciones dentro de diversas indrustrias.
No fue sino hasta que la empresa OpenAI lanzó la que es hasta hoy su producto estrella. El 30 de noviembre de 2022 fue  el día que el público general 
pudo experimentar, probar y comprender de manera más completa la gran revolución llamada inteligencia artificial generativa con la entrada a las masas de ChatGPT \cite{intro3},
implantando en la población el concepto de inteligencia artificial generativa.

De acuerdo con Google, ``La inteligencia artificial generativa se refiere al uso de la IA para crear contenido, como texto, imágenes, 
música, audio y video'' \cite{google1}. Ahora gracias a una interfaz amigable para todo público, resultó sencillo para personas de diversas industrias descubrir que existían 
diversas herramientas capaces de generar texto y responder preguntas de manera comprensible, incluso para aquellos que no eran expertos 
en el uso de este tipo de tecnologías y con ello también dar paso a la popularización de otros tipos de aplicaciones en donde 
la inteligencia artificial generativa también era prometedora, como Podría ser la generación de imágenes.

La génesis de esta tesis se basa en la experiencia de llevar a cabo un proyecto utilizando las tecnologías descritas previamente,
para que apartir de ella se pueda discutir sobre los riesgos asociados tanto de la creación como del uso de estas tecnologías. 
Bajo este contexto, entendemos el riesgo como la probabilidad de eventos adversos y su impacto debido al uso y desarrollo de sistemas de IA, 
abarcando desde fallas técnicas, ataques maliciosos hasta desafíos éticos y efectos socioeconómicos imprevistos \cite{intro1}. Este 
riesgo se consindera desde el desarrollo, hasta el uso de la aplicación creada.

El proyecto se centró en el uso de uno de los grandes modelo de lenguaje, conocido como LLM por sus siglas en inglés, limitándose a la su capacidad de 
generación de texto. Por lo tanto, no profundizaremos en otros tipos de inteligencia artificial generativa, como la 
generación de imágenes o audio. El enfoque principal de esta tesis se concentra solo en el área del procesamiento del lenguaje natural aplicados a LLM.

\newpage


\section{Problemática}

\par En Chile, se desarrollan numerosos proyectos en muchos sectores como pueden ser: la minería, inmobiliario, agrícola, salmonero, etc. Sin embargo, se ha observado un incremento en las reclamaciones 
sobre temas ambientales asociadas a estos proyectos, convirtiendo la ``Permisología ambiental'' en una problemática significativa para la generación de proyectos de toda índole dentro del país.

\par Muchos proyectos de inversión están judicializados, con sus causas llevadas ante el tribunal ambiental, debido a ello es crucial para las empresas poder prever estas reclamaciones y demandas 
potenciales con la mayor precisión posible \cite{p1}\cite{p2}\cite{p3}. Cada proyecto que se retrasa o cancela incrementa los costos y el riesgo, afectando no solo al proyecto en si, sino también la percepción del país como 
destino amistoso para inversiones extranjeras.

\par Además es necesarrio brindar a los directivos o personal pertinente un acceso inmediato y comprensible a información relevante de estos tribunales ambientales, superando las 
barreras del lenguaje legal que puede resultar complejo para quienes no son expertos en derecho. Esto les permitirá tener una visión general sobre posibles acciones, consecuencias y problemas que 
podrían surgir al tomar cualquier tipo de decisiones.

\section{Objetivos}
% Objetivos
\subsection{Objetivo General}
Reconocer los factores de riesgo asociados; tanto con la creación, como con el uso de aplicaciones que implementan Modelos 
Grandes de Lenguaje (LLM) en la industria, usando como caso de estudio el proyecto de búsqueda de jurisprudencia en tribunales ambientales. 

\subsection{Objetivos Específicos}

\begin{enumerate}
    \item Estructurar u organizar (un o el) contexto sobre inteligencia artificial, en específico la inteligencia artificial generativa y el procesamiento de lenguaje natural, que aborda esta Tesis. 
    \item Construir y desarrollar el estado del arte en lo que refiere a Modelos Grandes del Lenguaje y sus conceptos asociados.
    \item Analizar un proceso de ETL pertinente a la creación de la aplicación.
    \item Explicar el desarrollo de la aplicación asociada al chatbot, analizando los resultados obtenidos de este mismo.
    \item Reflexionar sobre los resultados obtenidos por el chatbot.
    \item Analizar y explicar los riesgos que conlleva la creación y uso de aplicaciones que utilizan Modelos Grandes de Lenguaje. 
    \item Analizar y concluir sobre cómo afrontar y el riesgo que tienen este tipo de tecnologías.
\end{enumerate}


\section{Metodología}

La metodología empleada en esta tesis se estructura en torno a cuatro componentes esenciales: 
el desarrollo del estado del arte, la creación del proyecto, los resultados asociados al proyecto y la evaluación de los riesgos asociados 
a cada etapa del proceso, tanto en la fase de desarrollo del proyecto como en la aplicación práctica del mismo.

\subsection{Estado del arte}
El Estado del Arte dará contexto sobre la inteligencia artificial, con un enfoque en los Modelos de Lenguaje de Gran Escala (LLM) 
y aplicaciones como ChatGPT. Se discutirá la evolución y el impacto de la inteligencia artificial, especialmente en el procesamiento 
del lenguaje natural (NLP) y la generación de contenido. También se explorará la arquitectura Transformer, clave para entender la 
eficacia de ChatGPT y otros modelos similares, y se abordará el proceso de ``embedding'' en el contexto de los modelos de lenguaje, entre otros.


\subsection{Creación del Proyecto}

Este capítulo aborda el desarrollo de un chatbot para consultar jurisprudencia en los Tribunales Ambientales de Chile, 
utilizando datos del ``Buscador Ambiental'' y la tecnología GPT-4 de OpenAI. Los pasos en su desarrollo son los siguientes:

\begin{itemize}
    \item \textbf{ETL - Extracción, Transformación y Carga:} Proceso clave para la preparación de datos.
    \begin{itemize}
        \item \textbf{Extracción:} Uso de Selenium y API para obtener datos del 'Buscador Ambiental'.
        \item \textbf{Transformación:} Conversión de PDF a TXT, y aplicación de técnicas de map-reduce.
        \item \textbf{Carga:} Almacenamiento de datos procesados en ChromaDB.
    \end{itemize}

    \item \textbf{Desarrollo del Chatbot:} Creación del chatbot utilizando Python, Flask y FastAPI.
    \begin{itemize}
        \item \textbf{Frontend:} Interfaz de usuario para realizar consultas.
        \item \textbf{Backend:} Integración con LangChain, uso de estructura RAG ,OpenAI GPT-4 para procesamiento y respuesta.
    \end{itemize}
\end{itemize}

\newpage

\subsection{Resultados}

En esta sección de la tesis, se evaluará el desempeño del chatbot creado previamente, se estudiará su funcionamiento y los resultados 
observando si estos coinciden con las indicaciones que se les entregó de manera previa. Se presentarán cuatro 
casos de estudio para analizar la calidad de respuestas según los diferentes prompts que se entregarán, abordando desde resúmenes de casos judiciales 
hasta solicitudes de información inexistente, lo cual revelará tanto las fortalezas como las limitaciones del sistema, 
especialmente en su capacidad para manejar la precisión y veracidad de la información.

\subsection{Evaluación de Riesgos}

Para terminar, previamente, se trabajará en la identificación y análisis de los potenciales riesgos  en cada una de las etapa del proceso, así como los riesgos derivados de
los resultados obtenidos en cada uno de los tipos de consulta. Esta evaluación de riesgo se estructura de la siguiente manera:

\begin{itemize}
    \item \textbf{Riesgos en la Creación del Proyecto:} Identificando los posibles obstáculos y contratiempos que pueden aparecer durante todas las etapas del desarrollo de una aplicación o herramienta que trabaje bajo los mismo tópicos que el chatbot realizado en esta tesis.
    \item \textbf{Riesgos en el Uso de la Aplicación:} Considerando de los riesgos asociados a la implementación práctica de proyectos con tecnologías parecidas o relacionadas a la desarrolladas en esta tesis.
\end{itemize}

Esta metodología proporciona un enfoque integral para la creación y aplicación de un proyecto utilizando grandes modelos de lenguaje (LLM), permitiendo una evaluación de los riesgos en cada etapa del proceso y en el caso específico de uso del chatbot creado para esta tesis. Esto facilita la toma de decisiones informadas y la formulación de estrategias para mitigar posibles contratiempos y dificultades previamente no estudiadas.

