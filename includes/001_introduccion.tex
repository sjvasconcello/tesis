
\chapter{Introducción}
La inteligencia artificial, también conocida como IA, ha experimentado un notable auge en la industria en los últimos tiempos
de la mano de la llamada Industria 4.0 \cite{intro1} y dentro del imaginario colectivo gracias a sus aplicacióny y facil acceso, 
especialmente en el ámbitos que durante mucho tiempo fueron algo reacias al cambio, como la administración y las finanzas \cite{intro2}. 
Este incremento no se debe necesariamente a un aumento en la capacidad cómputo a la que tenemos acceso, 
ya que esta capacidad ha ido creciendo gradualmente a lo largo del tiempo. Dicho lo anterior, aunque muy importante e investigado, 
la inteligencia artificial no generaba tanto interés como en la actualidad, porque esta estaba recervada para investigadores e implamentaciones dentro de diversas indrustrías.
No fue sino hasta que la empresa OpenAI lanzó la que es hasta hoy su producto estrella. El 30 de noviemnbre de 2022 fue  el dia que el público en general 
pudo experimentar, probar y comprender de manera más completa la gran revolución llamada inteligencia artificial generativa generativa con la entrada a las masas de ChatGPT \cite{intro3},
implantando en la población el concepto de inteligencia artificial generativa.

De acuerdo con Google, ``La inteligencia artificial generativa se refiere al uso de la IA para crear contenido, como texto, imágenes, 
música, audio y video'' \cite{google1}. Ahora gracias a una interfaz amigable para todo publico, resultó sencillo para personas de diversas industrias descubrir que existían 
diversas herramientas capaces de generar texto y responder preguntas de manera comprensible, incluso para aquellos que no eran expertos en 
en el uso de este tipo de tecnologías y con ello también dar paso a la popularización de otros tipos de aplicaciones en donde 
la inteligencia artificial generativa tambien era prometedora, como podria ser la generación de imagenes.

La génesis de esta tesis se basa en la experiencia de llevar a cabo un proyecto utilizando las tecnologías descrita previamente,
para que apartir de ella se pueda discutir sobre los riesgos asociados tanto de la creación como del uso de esatas tecnologías. 
Bajo este contexto, entendemos el riesgo como la probabilidad de eventos adversos y su impacto debido al uso y desarrollo de sistemas de IA, 
abarcando desde fallas técnicas, ataques maliciosos hasta desafíos éticos y efectos socioeconómicos imprevistos \cite{intro1}. Este 
riesgo se consindera desde el desarrollo y el uso de la aplicacion creada.

El proyecto se centró en el uso de uno de los grandes modelo de lenguaje, conocido como LLM por sus siglas en inglés, limitándose a la su capacidad de 
generación de texto. Por lo tanto, no profundizaremos en otros tipos de inteligencia artificial generativa, como la 
generación de imágenes o audio. El enfoque principal de esta tesis se concentra solo en el area del procesamiento del lenguaje natural aplicados a LLM.

\newpage

\section{Obejetivos}
% Objetivos
\subsection{Objetivo General}
Esta tesis se enfocará principalmente en identificar los factores de riesgo asociados tanto con la creación como con el 
uso de aplicaciones que implementan Grande Modelos de Lenguaje (LLM) en la industria usando como caso de 
estudio el proyecto de búsqueda de jurisprudencia en tribunales ambientales.

\subsection{Objetivo Específico}

\begin{enumerate}
    \item Entregar contexto sobre inteligencia artificial y el area en específico que aborda esta Tesis.
    \item Desarrollar el estado del arte en lo que refiere a Modelos Grandes del Lenguaje y sus conceptos asociados
    \item Analizar un proceso de ETL pertinente a la creación de la aplicación
    \item Explicar el desarrollo de la aplicación asociada al chatbot
    \item Comentar los resultados obtenidos de este chatbot
    \item Estudiar los riesgos que conlleva la creación y uso de aplicaciones que utilizan Modelos Grandes de Lenguaje
    \item Concluir sobre como afrontar y el riesgo que tienen este tipo de tecnología
\end{enumerate}


\section{Metodologia}

La metodología empleada en esta tesis se estructura en torno a cuatro componentes esenciales: 
el desarrollo del estado del arte, la creación del proyecto, los resultados asociados al proyecto y la evaluación de los riesgos asociados 
a cada etapa del proceso, tanto en la fase de desarrollo del proyecto como en su aplicación práctica del mismo.

\subsection{Estado del arte}
El Estado del Arte dará contexto sobre la inteligencia artificial, con un enfoque en los Modelos de Lenguaje de Gran Escala (LLM) 
y aplicaciones como ChatGPT. Se discutira la evolución y el impacto de la inteligencia artificial, especialmente en el procesamiento 
del lenguaje natural (NLP) y la generación de contenido. También se explora la arquitectura Transformer, clave para entender la 
eficacia de ChatGPT y otros modelos similares, y se aborda el proceso de ``embedding'' en el contexto de los modelos de lenguaje, entre otros.


\subsection{Creación del Proyecto}


\noindent Esta fase inicial comprende el desarrollo del proyecto basado en IA generativa. Incluye los siguientes pasos:

\begin{itemize}
    \item \textbf{Definición de Objetivos y Alcance:} Establecimiento claro de los propósitos y límites del proyecto, identificando las metas a alcanzar.
    \item \textbf{Selección de Tecnologías y Herramientas:} Evaluación y elección de las tecnologías y herramientas apropiadas para la implementación del proyecto.
    \item \textbf{Diseño de la Arquitectura:} Desarrollo de la estructura y componentes del proyecto, considerando aspectos de escalabilidad y rendimiento.
    \item \textbf{Implementación y Desarrollo:} Construcción efectiva del proyecto, incluyendo la programación y configuración de la inteligencia artificial generativa.
\end{itemize}
\subsection{Ejemplo de Uso del Proyecto}

Esta fase implica la aplicación práctica del proyecto en un contexto específico, demostrando su funcionalidad y utilidad. 
En este caso nuestro interés mas que en el output que genere la aplicación, es como funciona internamente el proceso cosa 
que para la siguiente etapa sea más fácil 

\subsection{Evaluación de Riesgos}

Se trabajará la identificación y análisis de los riesgos potenciales en cada etapa del proceso, así como los riesgos derivados
del caso de uso. Incluye:

\begin{itemize}
    \item \textbf{Riesgos en la Creación del Proyecto:} Identificación de posibles obstáculos y contratiempos durante la etapa de concepción y desarrollo.
    \item \textbf{Riesgos en el Uso de la Aplicación:} Consideración de los riesgos asociados a la implementación práctica del proyecto en el contexto definido.
\end{itemize}

Esta metodología proporciona un enfoque integral para la creación y aplicación de un proyecto utilizando LLM, permitiendo una evaluación de los riesgos en cada etapa del proceso y en el caso de uso específico. Esto facilita la toma de decisiones informadas y la formulación de estrategias para mitigar posibles contratiempos.

\subsection{Problematica}

\par En Chile, se desarrollan numerosos proyectos en muchos sectores como pueden ser: la minería, inmobiliario, agrícola, salmonero, etc. Sin embargo, se ha observado un incremento en las reclamaciones 
sobre temas ambientales asociadas a estos proyectos, convirtiendo la ``Permisología ambiental'' en una problemática significativa para la generación de proyectos de toda índole dentro del país.

\par Muchos proyectos de inversión están judicializados, con sus causas llevadas ante el tribunal ambiental, debido a ello es crucial para las empresas poder prever estas reclamaciones y demandas 
potenciales con la mayor precisión posible \cite{p1}\cite{p2}\cite{p3}. Cada proyecto que se retrasa o cancela incrementa los costos y el riesgo, afectando no solo al proyecto en sí, sino también la percepción del país como 
destino amistoso para inversiones extranjeras.

\par La propuesta presentada esta tesis es brindar a los directivos o personal pertinente un acceso inmediato y comprensible a información relevante de estos tribunales ambientales, superando las 
barreras del lenguaje legal que puede resultar complejo para quienes no son expertos en derecho. Esto les permitirá tener una visión general sobre posibles acciones, consecuencias y problemas que 
podrían surgir al tomar cualquier tipo  dedecisiones.

\par En conclusión, esta tesis, que se enfoca en estudiar los riesgos asociados a los proyectos, junto con el desarrollo de un chatbot alimentado con datos de las reclamaciones del tribunal 
ambiental, aborda de manera efectiva la problemática identificada.


