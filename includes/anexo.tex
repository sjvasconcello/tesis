%!TEX root = memoria.tex
\chapter{Contexto enviado por el Chatbot}\label{appx:output}

\begin{Verbatim}[frame=lines, label=Elavoración propia -  Ejemplo de Contexto enviado por el Chatbot a OpenAI
				, fontsize=\footnotesize, numbers=left
				, baselinestretch=0.4
				, formatcom=\color{gray}]
{
    'query': 'Dame un resumen del caso Dominga',
    'result': 'El caso Dominga se refiere a una reclamación presentada por Andes Iron SpA
contra la Resolución Exenta N° 1146/2017 del Servicio de Evaluación Ambiental y el Comité de
Ministros. El reclamo se basa en el rechazo del Estudio de Impacto Ambiental del proyecto
Dominga y la acogida parcial de los recursos de reclamación presentados en el marco de la
participación ciudadana. El Primer Tribunal Ambiental de la República de Chile admitió la
reclamación y convocó a una audiencia de percepción documental. Además, se presentaron
informes de Amicus Curiae y se aceptó la participación de terceros coadyuvantes e
independientes. El proceso judicial continúa y se espera una sentencia definitiva. En una
sentencia anterior, el tribunal había anulado varias resoluciones relacionadas con el proyecto
y ordenado retrotraer el procedimiento de evaluación ambiental a una etapa posterior al
Informe Consolidado de Evaluación. Sin embargo, esta sentencia fue anulada por la Corte
Suprema y el caso fue remitido nuevamente al Primer Tribunal Ambiental para que se pronuncie
sobre el fondo del asunto.',
    'source_documents': [
        Document(
            page_content='El caso Dominga\n\nRol: "R-1-2017", Rol: "R-1-2017",Rol: "R-1-2017",
fecha 16-04-2021 de Andes Iron SpA con SEA cuyo proyecto corresponde a Dominga. Este caso
tiene como principal materia Evaluación Ambiental y su coponente ambiental es Agua\n\nThe
integration ministers are: \n1. Juan Opazo Lagos\n2. Jasna Pavlich Núñez\n3. Fabrizio Queirolo
Pellerano\n\nSiendo sus descriptores: Evaluación ambiental, término anticipado, suficiencia de
información, medidas de compensación, valor compartido, facultades del Comité de Ministros,
suficiencia de la información, línea base de medio marino, medida voluntaria, riesgos
ambientales, observación ciudadana, PAC, fauna marina, impacto de ruidos, rutas de navegación,
efectos sinérgicos, áreas de exclusión, trato desigual, biodiversidad marina, pingüino
Humboldt, barreras eólicas, amicus curiae. \n\nEl First Environmental Court Acoge la causa en
esta ocacion. Los hechos controvertidos son: Del supuesto carácter completo de los
antecedentes del proyecto y evaluación del mismo. En cuanto al término anticipado del
procedimiento de evaluación ambiental. En cuanto al criterio del propio Comité de Ministros
respecto de los aspectos relacionados con la actividad de transporte marítimo. De la supuesta
información para evaluar las rutas de navegación que habría cumplido con creces el estándar
fijado por el Comité de Ministros. En relación a la propuesta de la casa del Mar, luego
denominada centro de estudios científicos. En cuanto al “Valor Compartido” con las
comunidades. Revisión de los argumentos técnicos que fundaron el rechazo del proyecto. De la
supuesta insuficiencia de la línea de base del medio marino en las rutas de navegación. De la
supuesta utilización parcial de fragmentos contenidos en los informes de los Servicios, con
preguntas inductivas al efecto. Del supuesto tratamiento discriminatorio en lo relativo a la
línea de base. De la supuesta suficiencia de los análisis de efectos sinérgicos. De la
supuesta ausencia de desequilibrio entre impactos y medidas. De las Reclamaciones PAC. Del
supuesto impacto en el recurso hídrico luego del cierre del Sistema de Captación y Devolución
de aguas desde el Rajo Sur. De la supuesta claridad de la información para descartar impactos
adversos en el mar. De la supuesta insignificancia de impactos por aporte de material
particulado sedimentable (“MPS”). De la supuesta insuficiencia de la medida de compensación
para el impacto en el proceso de nidificación de pingüino de Humboldt. De la supuesta
ilegalidad en la posición sobre la evaluación del impacto de barreras eólicas en la avifauna.
De la supuesta suficiencia de las áreas de exclusión. De la supuesta evaluación del impacto
acústico de los buques en la fauna marina. Del supuesto plan de monitoreo de aves y cetáceos
que no podría supuestamente ser medida.\n\nY el resumen es: El Primer Tribunal Ambiental de la
República de Chile recibió una reclamación presentada por Andes Iron SpA contra la Resolución
Exenta N° 1146/2017 del Servicio de Evaluación Ambiental y el Comité de Ministros. La
reclamación se basa en el rechazo del Estudio de Impacto Ambiental del proyecto Dominga y la
acogida parcial de los recursos de reclamación presentados en el marco de la participación
ciudadana. El Tribunal admitió a trámite la reclamación y convocó a una audiencia de
percepción documental. Además, se presentaron informes de Amicus Curiae y se aceptó la
participación de terceros coadyuvantes e independientes. El proceso judicial continúa y se
espera una sentencia definitiva.\nEl Primer Tribunal Ambiental de Chile ha emitido una
sentencia en relación al proyecto minero-portuario Dominga. En la sentencia, se rechaza la
excepción de extemporaneidad interpuesta por un tercero independiente y se acoge la
reclamación presentada por Andes Iron SpA. Se anulan varias resoluciones relacionadas con el
proyecto y se ordena retrotraer el procedimiento de evaluación ambiental a una etapa posterior
al Informe Consolidado de Evaluación. La Corte Suprema ha anulado la sentencia y ha remitido
el caso nuevamente al Primer Tribunal Ambiental para que se pronuncie sobre el fondo del
asunto.\nEl documento revisado es una serie de antecedentes procesales de un caso relacionado
con el proyecto minero Dominga en Chile. El caso involucra reclamaciones judiciales y recursos
de casación presentados por diferentes partes. La Corte Suprema ha ordenado al Tribunal
Ambiental que revise y decida sobre el fondo del asunto, incluyendo la evaluación ambiental
del proyecto y la refutación de los antecedentes técnicos utilizados para rechazarlo. Los
principales hechos controvertidos incluyen el carácter completo de los antecedentes del
proyecto, la evaluación del mismo y los argumentos técnicos que fundamentaron su rechazo. El
documento también menciona la suspensión de la vista de la causa y la fijación de una nueva
fecha para la audiencia de alegatos.\nEl Tribunal Ambiental de Chile revisó una serie de
alegaciones relacionadas con la evaluación ambiental del proyecto Dominga. En primer lugar, se
discutió si el procedimiento de evaluación se terminó anticipadamente, concluyendo que no
existió falta de información relevante. Además, se argumentó que el proyecto cumplió con todas
las instancias formales de participación ciudadana. En segundo lugar, se analizó el criterio
del Comité de Ministros respecto a la actividad de transporte marítimo del proyecto,
concluyendo que el Comité consideró que la línea de base complementaria para las rutas de
navegación no fue suficiente para evaluar los efectos del proyecto. Sin embargo, el Tribunal
consideró que la línea de base del medio marino presentada por el titular del proyecto fue
suficiente y adecuada. Por último, se discutió la propuesta de la Casa del Mar, concluyendo
que el proyecto consideró compromisos voluntarios para cumplir con estándares más altos que
los establecidos por la legislación ambiental. En resumen, el Tribunal acogió algunas
alegaciones del reclamante y desestimó otras, concluyendo que el proyecto Dominga cumplió con
los requisitos de evaluación ambiental.\nEl Tribunal Ambiental ha revisado los argumentos
técnicos que fundamentaron el rechazo del proyecto Dominga. En cuanto a la medida propuesta de
crear un Centro de Estudios Científicos, el tribunal considera que es interesante e
innovadora, pero sugiere que debería ser más amplia en su alcance y concepción, abarcando
también el estudio de los ecosistemas terrestres y el medio humano en la comuna de La Higuera.
Además, se propone que el Centro de Estudios lleve a cabo un programa de investigación y
estudios de las especies de interés en la zona, así como un protocolo de acción frente a
impactos ambientales no previstos. También se plantea la necesidad de medir los posibles
impactos socioeconómicos en las comunidades pesqueras, turísticas y productivas de Los Choros
y Chungungo, y elaborar un plan de desarrollo en armonía con el plan de desarrollo regional.
El tribunal destaca la importancia de establecer un mecanismo vinculante para las decisiones
del Centro de Estudios y la participación de actores relevantes en su diseño y gobierno
corporativo. En cuanto al "Valor Compartido" con las comunidades, el tribunal considera que es
importante que los proyectos entreguen beneficios compartidos y reconoce la propuesta de
Dominga en este sentido. Sin embargo, señala que esta materia no se encuentra discutida en el
proceso de evaluación y que el compromiso de compartir el valor del negocio no puede sobreseer
las causales por las cuales el proyecto fue rechazado. En relación a la supuesta insuficiencia
de la línea de base del medio marino en las rutas de navegación, el tribunal considera que la
información presentada por el proyecto es suficiente y que el Comité de Ministros no hizo un
análisis adecuado al respecto. Se destaca que el compendio de conocimiento entregado por el
proyecto es robusto y que la campaña de terreno realizada en las rutas de navegación
complementa adecuadamente la línea de base del medio marino. En conclusión, el tribunal acoge
los cuestionamientos del reclamante y mantiene el rechazo del proyecto Dominga.\nEl Tribunal
Ambiental de Chile revisó una alegación presentada por el Titular del proyecto Dominga, en la
que se argumentaba que el Comité de Ministros había aplicado criterios diferentes al evaluar
los proyectos Dominga y Cruz Grande en relación a la actividad de navegación. El Titular
sostenía que ambos proyectos eran similares y debían ser evaluados de manera análoga para
respetar el principio de igualdad ante la ley. Sin embargo, el Tribunal determinó que los
proyectos no se encontraban en la misma situación jurídica o material, por lo que era
razonable aplicar criterios diferentes. Además, se señaló que el proyecto Dominga no había
evaluado adecuadamente los impactos de la navegación, a pesar de haber identificado riesgos y
presentado planes de contingencia. En conclusión, el Tribunal rechazó la alegación de trato
discriminatorio y confirmó la decisión del Comité de Ministros.\nEl Tribunal Ambiental ha
analizado los elementos de temporalidad, objetivos y ubicación de los proyectos Cruz Grande y
Dominga. Ambos proyectos fueron presentados a través de Estudios de Impacto Ambiental y fueron
tramitados en forma paralela durante tres años. Ambos proyectos se ubican en la comuna de La
Higuera, Región de Coquimbo. Ambos proyectos tienen como objetivo la recepción, acopio y
embarque de sustancias minerales, por lo que es necesario evaluar los impactos ambientales
asociados a estas actividades. El Comité de Ministros adoptó un actuar distinto respecto al
proyecto Dominga, argumentando la insuficiencia de información sobre las rutas de navegación y
los impactos que podrían generar. Sin embargo, el Tribunal considera que el Comité de
Ministros contaba con la información necesaria para evaluar adecuadamente el proyecto Dominga.
En cuanto a los impactos sinérgicos, el Tribunal concluye que se realizó un análisis completo
de estos impactos en el proyecto Dominga, considerando la existencia del proyecto Cruz Grande.
El Tribunal considera que la información de línea base es suficiente para definir los impactos
sinérgicos y que no existe un desequilibrio entre los impactos y las medidas propuestas.\nEl
Tribunal Ambiental ha revisado las reclamaciones presentadas en relación al proyecto minero
Dominga. En cuanto a la insuficiencia de la línea de base del medio marino en las rutas de
navegación, el Tribunal concluye que no está debidamente acreditada y que las medidas
asociadas son adecuadas. En relación al impacto en el recurso hídrico por el cierre del
Sistema de Captación y Devolución de aguas desde el Rajo Sur, el Comité de Ministros consideró
que la medida de mitigación era inadecuada, pero el SEA argumenta que la medida es idónea y
que se puede financiar a través del Fondo para la Gestión de Faenas Mineras Cerradas. Respecto
a los efectos en el mar, el Comité de Ministros concluyó que no se evaluaron adecuadamente los
impactos de la planta desalinizadora, pero el SEA argumenta que sí se evaluaron y que no se
evidencian cambios significativos en las condiciones físico-químicas ni en la biodiversidad
asociada a las Áreas de Manejo y Explotación de Recursos Bentónicos. Además, se establece la
necesidad de un monitoreo de la salinidad durante la operación del proyecto. En conclusión, el
Tribunal rechaza las reclamaciones presentadas.\nEl Tribunal Ambiental concluye que las
preocupaciones presentadas por la empresa Dominga en relación a las preocupaciones PAC
(Programa de Adecuación y Compensación) fueron debidamente resueltas. Sin embargo, se
determina que hubo una subestimación en la evaluación de los impactos por aporte de material
particulado sedimentable (MPS) y se evidencia un desconocimiento de los antecedentes por parte
del Comité de Ministros. Además, se señala que la normativa ambiental no se adecúa al contexto
del proyecto. En cuanto a la medida de compensación para el impacto en el proceso de
nidificación del Pingüino de Humboldt, se determina que la propuesta presentada por la empresa
no logra un efecto positivo alternativo y equivalente, por lo que se acoge la reclamación. Se
cuestiona la falta de fundamentos del Comité de Ministros y se sugiere la implementación de
medidas de mitigación más innovadoras.\nEl Tribunal Ambiental advierte una debilidad en la
forma en que se abordó la afectación del Pingüino de Humboldt en el islote Totoralillo Norte
durante la evaluación ambiental del proyecto. Se señala que no se consideraron medidas de
mitigación innovadoras ni propuestas alternativas de medidas compensatorias. Además, se
cuestiona la medida de compensación propuesta, ya que no cumple con los requisitos de
equivalencia y adicionalidad. En cuanto al impacto de las barreras eólicas en la avifauna, se
argumenta que este tema no fue debidamente considerado ni evaluado durante la evaluación
ambiental del proyecto. Se destaca la falta de congruencia y fundamentación en la resolución
del Comité de Ministros. Por otro lado, se menciona la importancia de evaluar el efecto
atrayente de las luminarias de las barreras eólicas y la infraestructura portuaria en la
avifauna marina. Se hace referencia a la contaminación lumínica y sus efectos desorientadores
en las aves costeras.\nLa revisión del expediente de evaluación ambiental no muestra estudios
relacionados con la elaboración de una línea base de la avifauna en los sectores mencionados,
lo que dificulta la identificación y abordaje de los impactos potenciales. El uso de
luminarias durante la noche en zonas portuarias y la mortalidad de aves marinas, especialmente
los petreles, son temas sensibles que deben ser considerados. Se advierte la oportunidad de
complementar la información técnica para evaluar adecuadamente estos impactos. Además, se
cuestiona la supuesta suficiencia de las áreas de exclusión y la falta de consideración de los
impactos sinérgicos. Se destaca la falta de estudios sobre el impacto acústico de los buques
en la fauna marina. En conclusión, se considera que no se evaluaron adecuadamente los impactos
mencionados y se sugiere la realización de investigaciones científicas para incrementar el
conocimiento de los ecosistemas marino costeros.\nEl Primer Tribunal Ambiental de Chile ha
acogido la reclamación presentada por Andes Iron SpA en contra de la Resolución 1146 del
Director Ejecutivo del SEA y el acuerdo Nº 08/2017 del Comité de Ministros. El tribunal ha
ordenado retrotraer el procedimiento de evaluación ambiental a la etapa posterior al Informe
Consolidado de Evaluación Ambiental y realizar una nueva votación ajustada a derecho por parte
de la COEVA de la Región de Coquimbo. No se condena en costas a las partes.\nResumen:\n\nEl 16
de abril de 2021, se notificó en Antofagasta la sentencia anterior a través del estado diario
y correo electrónico. La sentencia se encuentra en las fojas 6156.',
            metadata={
                'caratula': 'Andes Iron SpA con SEA',
                'competencia': 'Art. 17 N° 5 de Ley N° 20.600',
                'componente_ambiental': 'Agua',
                'decision_del_tribunal': 'Acoge',
                'descriptores': 'Titular del proyecto; vicios formales; motivaciòn; nulidad;
vicios de procedimiento; plazos; votación;debido proceso; garantías; contradictoriedad;
congruencias; igualdad ante la ley; medio marino',
                'fechaSentencia': '27-04-2018',
                'nombreProyecto': 'Dominga',
                'page': 0,
                'rol': 'R-1-2017',
                'source':
'/home/sjvasconcello/usm-data/ds-senior/cmad-ai-sustentable-01/etl/load/../../data/clean/docs_
with_summary/R-1-2017.txt',
                'tribunal': '1TA',
                'urlSentencia':
'https://www.buscadorambiental.cl/buscador-api/file/view/jurisprudencia/S1TA Rol N°
R-1-2017?id=1'
            }
        ),
        Document(
            page_content='El caso Caserones\n\nRol: "R-41-2021", Rol: "R-41-2021",Rol:
"R-41-2021", fecha 31-08-2021 de Asociación de Productores y Exportadores Agrícolas del Valle
de Copiapó y otros con Superintendencia de Medio Ambiente cuyo proyecto corresponde a Proyecto
Caserones. Este caso tiene como principal materia Programa de cumplimiento SMA y su coponente
ambiental es Agua\n\nThe integration ministers are: \n1. Mauricio Oviedo Gutiérrez\n2. Juan
Opazo Lagos\n3. Marcelo Hernández Rojas\n\nSiendo sus descriptores: Procedimiento
sancionatorio, programa de cumplimiento, SMA, desagregación de cargos, desviación del
procedimiento, integridad del programa de cumplimiento, procedencia del PDC, infracciones que
han causado daño ambiental, criterio de integridad, plan de reparación ambiental, incentivos
al cumplimiento, falta de perjuicio, principio de confianza legitima, calidad de aguas,
acuíferos, criterio de eficacia, razonabilidad y motivación de la resolución de la SMA,
idoneidad de las medidas, Guía para la presentación de Programas de Cumplimiento por
infracciones a instrumentos de carácter ambiental.\n\nEl First Environmental Court Rechaza la
causa en esta ocacion. Los hechos controvertidos son: Procedencia del Programa de Cumplimiento
para infracciones que han causado daño ambiental, Sobre el criterio de eficacia del Programa
de Cumplimiento y la prohibición general de eludir a la responsabilidad.\n\nY el resumen es:
En este documento se presenta una revisión de un caso judicial en el que la Asociación de
Productores y Exportadores Agrícolas del Valle de Copiapó, la Comunidad de Aguas Subterráneas
Sector N° 1 "Aguas Arriba del Embalse Lautaro", la Comunidad de Aguas Subterráneas Sector N°2
"Embalse Lautaro - La Puerta" y la Comunidad de Aguas Subterráneas Sector N° 3 "La Puerta -
Mal Paso" presentaron una reclamación contra la Resolución Exenta Nº 15/Rol D-018-2019 de la
Superintendencia del Medio Ambiente. En la reclamación, solicitan la invalidación de la
resolución y el rechazo del programa de cumplimiento presentado por SCM Minera Lumina Copper
Chile S.A. El caso involucra cargos relacionados con el incumplimiento de condiciones
ambientales establecidas para el proyecto Caserones. La Superintendencia del Medio Ambiente y
SCM Minera Lumina Copper Chile S.A. presentaron sus argumentos en contra de la reclamación. El
Tribunal admitió la reclamación a trámite y rechazó la solicitud de medida cautelar presentada
por los reclamantes. El caso se encuentra en estado de relación y se llevó a cabo una vista el
5 de mayo de 2021.\nLa reclamante argumenta que la resolución impugnada infringe la ley al
desagregar dos de los cargos formulados en la Resolución Exenta N° 1/D-018-2019, lo cual
atenta contra la finalidad de poner término al procedimiento administrativo sancionador a
través del programa de cumplimiento. Además, sostiene que la desagregación de los cargos
infringe el criterio de integridad y que la resolución reclamada carece de una causal legal
que habilite la desagregación del procedimiento administrativo. Por otro lado, la reclamante
acusa a la resolución de violar el principio de confianza legítima al desagregar el
procedimiento sin causa legal alguna. En cuanto a los argumentos de la reclamada, esta
sostiene que la resolución reclamada es legal y fue dictada conforme a la normativa vigente.
Además, argumenta que la empresa ha acreditado la no generación de efectos negativos en
relación al recurso hídrico y que las medidas propuestas en el programa de cumplimiento son
eficaces para contener la pluma de contaminación. También señala que la empresa ha cumplido
con las acciones de difusión y sociabilización del pueblo Colla, pese a no haber sido
coordinadas con la CONADI. En relación al cargo N° 9, la reclamada sostiene que la empresa
pudo acreditar la no generación de efectos negativos y que el caudal entregado de agua
desalada permitió mejorar la calidad del agua entregada a la población. Por último, la
reclamada argumenta que la improcedencia del programa de cumplimiento en casos de daño
ambiental se debe a las características de las infracciones y al contexto regulatorio
ambiental en que se inserten.\nEl Programa de Cumplimiento (PDC) no es un derecho del titular
que deba operar en todo momento, ya que solo cumple su objetivo a través de medidas que
permitan volver al cumplimiento de la normativa ambiental. La desagregación de los cargos 11 y
12 por parte de la Superintendencia del Medio Ambiente (SMA) cumple con la finalidad del
procedimiento sancionatorio de obtener la responsabilidad del infractor y proteger el medio
ambiente. La improcedencia del PDC respecto a estos cargos no genera perjuicios a los
reclamantes, ya que el derecho a presentar un PDC corresponde únicamente al presunto infractor
y no a los interesados del procedimiento. La SMA argumenta que la desagregación de los cargos
11 y 12 está en armonía con los intereses de los denunciantes. Además, la SMA se refiere a la
Resolución Exenta N° 7/Rol D-011-2015 para justificar su actuar y sostiene que ha sido
consistente con la práctica del Servicio. Por otro lado, las reclamantes argumentan que la
desagregación de los cargos 11 y 12 vulnera el principio de integridad y que no existe una
norma legal que justifique esta desagregación. También sostienen que el PDC y los Planes de
Reparación Ambiental (PRA) son mecanismos compatibles y que el cumplimiento satisfactorio de
un PDC constituye un equivalente procesal al procedimiento sancionatorio. La SMA argumenta que
el PDC no es procedente para infracciones que han causado daño ambiental, ya que no existen
acciones que puedan recomponer el medio ambiente dañado a través de este instrumento. Además,
señala que el PDC no es un derecho que deba operar a todo evento y que la desagregación de los
cargos 11 y 12 no genera perjuicios a los reclamantes. En cuanto al principio de confianza
legítima, la SMA sostiene que su actuar ha sido consistente con sus propios precedentes. En
resumen, las controversias se centran en la procedencia del PDC para infracciones que han
causado daño ambiental y en la compatibilidad entre el PDC y los PRA. La SMA argumenta que el
PDC no es procedente para este tipo de infracciones, mientras que las reclamantes sostienen lo
contrario.\nEl documento revisa la aplicación de un Programa de Cumplimiento (PDC) y un Plan
de Reparación Ambiental (PRA) en el contexto de un procedimiento administrativo sancionatorio
por infracciones ambientales. Se destaca que el PDC es un instrumento voluntario de incentivo
al cumplimiento normativo, mientras que el PRA se aplica en casos de infracciones con daño
ambiental. Se argumenta que la desagregación de los cargos 11 y 12, que generó la aprobación
de un PDC y la continuación del procedimiento para el PRA, no vulnera el criterio de
integridad y no produce perjuicio a los reclamantes. Además, se señala que la decisión de
desagregación se ajusta a la ley y no afecta el principio de confianza legítima.\nEl artículo
revisa el principio de confianza legítima en la actuación de la administración como límite a
la potestad invalidatoria. Se analiza la actuación de la Superintendencia del Medio Ambiente
(SMA) en casos similares para determinar si se ha infringido este principio. Se destaca que la
SMA ha establecido un criterio expreso para el caso de infracciones con daño ambiental y su
incompatibilidad con el Programa de Cumplimiento (PdC). Se mencionan casos en los que la SMA
ha aprobado o rechazado PdC de acuerdo a este criterio. Se discute la eficacia de las medidas
incluidas en el PdC y se concluye que la SMA ha realizado un análisis razonable y motivado
para descartar los efectos negativos de las infracciones. Se rechaza la reclamación presentada
por las asociaciones agrícolas y de aguas subterráneas. Se hace un llamado a la SMA para
agilizar la revisión de los cargos desagregados y tomar medidas oportunas en el procedimiento
administrativo sancionador en trámite.\nEl Primer Tribunal Ambiental de Chile emitió una
sentencia en el caso de la Quebrada La Brea y el Río Ramadillas. En la sentencia, se establece
que la empresa LOSMA debe presentar un "Plan de Reparación" por los daños ambientales causados
y, en caso de no ser posible, los antecedentes serán derivados al Consejo de Defensa del
Estado para que ejerza la acción judicial correspondiente. El tribunal destaca la importancia
de garantizar el derecho a vivir en un medio ambiente libre de contaminación y el deber del
Estado de preservar los recursos hídricos. La sentencia fue redactada por el Ministro Mauricio
Oviedo Gutiérrez y firmada por los Ministros Marcelo Hernández Rojas y Juan Opazo Lagos. El
Secretario Abogado del Tribunal, Pablo Miranda Nigro, autorizó la notificación de la
sentencia.',
            metadata={
                'caratula': 'Asociación de Productores y Exportadores Agrícolas del Valle de
Copiapó y otros con Superintendencia de Medio Ambiente',
                'competencia': 'Art. 17 N° 3 de Ley N° 20.600',
                'componente_ambiental': 'Agua',
                'decision_del_tribunal': 'Rechaza',
                'descriptores': 'Procedimiento sancionatorio; programa de cumplimiento; SMA;
desagregación de cargos; desviación del procedimiento; integridad del programa de
cumplimiento; procedencia del PDC; infracciones que han causado daño ambiental; criterio de
integridad; plan de reparación ambiental; incentivos al cumplimiento; falta de perjuicio;
principio de confianza legitima; calidad de aguas; acuíferos; criterio de eficacia;
razonabilidad y motivación de la resolución de la SMA; idoneidad de las medidas; Guía para la
presentación de Programas de Cumplimiento por infracciones a instrumentos de carácter
ambiental.',
                'fechaSentencia': '31-08-2021',
                'nombreProyecto': 'Proyecto Caserones',
                'page': 0,
                'rol': 'R-41-2021',
                'source':
'/home/sjvasconcello/usm-data/ds-senior/cmad-ai-sustentable-01/etl/load/../../data/clean/docs_
with_summary/R-41-2021.txt',
                'tribunal': '1TA',
                'urlSentencia':
'https://www.buscadorambiental.cl/buscador-api/file/view/jurisprudencia/S1TA Rol N°
R-41-2021?id=28'
            }
        ),
        Document(
            page_content='El caso Escombrera Palguin bajo\n\nRol: "R-25-2019", Rol:
"R-25-2019",Rol: "R-25-2019", fecha 24-03-2020 de Elsa Quirquitripay Antiman y Otros con
SEREMI de Salud Región de la Araucanía cuyo proyecto corresponde a Escombrera, sitio de
disposición de residuos no peligrosos. Este caso tiene como principal materia Invalidación
administrativa y su coponente ambiental es Agua\n\nThe integration ministers are: \n1. Iván
Hunter Ampuero\n2. Jorge Retamal Valenzuela\n3. Sibel Villalobos Volpi\n\nSiendo sus
descriptores: Competencia absoluta, instrumento de gestión ambiental, legitimación activa,
acción ,invalidación facultad,  invalidación impropia \n\nEl Third Environmental Court Rechaza
la causa en esta ocacion. Los hechos controvertidos son: Si el Tribunal tendría competencia
para pronunciarse sobre la reclamación judicial, considerando la naturaleza de las
resoluciones impugnadas, Si los Reclamantes tendrían legitimación activa para ejercer
legalmente la impugnación judicial\n\nY el resumen es: El abogado Marco Antonio Román Cordero,
en representación de ACONSER RESIDUOS SPA, interpuso una reclamación contra la
SUPERINTENDENCIA DEL MEDIO AMBIENTE (SMA) por la Resolución Ex. N° 427, que requería a la
empresa el ingreso del proyecto "Vertedero Aconser Mocopulli" al Sistema de Evaluación de
Impacto Ambiental (SEIA). La reclamante solicitó al tribunal acoger su reclamación y declarar
la ilegalidad de la resolución reclamada. Por su parte, la SMA solicitó al tribunal rechazar
la reclamación y declarar la legalidad de la resolución. En los antecedentes del acto
administrativo reclamado, se mencionan diversos documentos y comunicaciones entre la SEREMI de
Salud, la SMA y la Reclamante. En cuanto al procedimiento de reclamación, se detallan las
alegaciones de ambas partes y los argumentos presentados. La Reclamante argumenta que el
procedimiento de requerimiento de ingreso se inició de manera irregular y que se violaron
plazos establecidos por la ley. Además, cuestiona la competencia de la SMA para dictar la
resolución reclamada. La SMA, por su parte, argumenta que la documentación relevante está
incluida en el expediente administrativo y que la Reclamante tuvo acceso a la información
necesaria. También se menciona que el desconocimiento de ciertos documentos no supone
indefensión, ya que la SMA constató los mismos hallazgos comunicados por la SEREMI de Salud.
En conclusión, la Reclamante solicita la anulación o modificación de la resolución reclamada,
mientras que la SMA pide que se declare su legalidad.\nLa reclamación presentada por ACONSER
Residuos SpA no cuestiona el cumplimiento de los supuestos fácticos y jurídicos contenidos en
la resolución impugnada. Sin embargo, se plantean controversias relacionadas con el
procedimiento administrativo que precedió a la dictación del acto terminal de requerimiento de
ingreso al SEIA. Estas controversias incluyen la falta de conocimiento de la denuncia que
originó el requerimiento, la demora excesiva en resolver el procedimiento, la omisión en la
aplicación del Instructivo para la Tramitación de Requerimientos de Ingreso al SEIA y la
omisión en decretar diligencias probatorias solicitadas por la interesada. Respecto a la falta
de conocimiento de la denuncia, se determina que el expediente administrativo físico contiene
el procedimiento que justifica la dictación del acto, por lo que la tardanza en incorporar la
denuncia al sistema electrónico no constituye un vicio de legalidad. En cuanto a la demora en
resolver el procedimiento, se establece que el plazo de 6 meses establecido por la ley no se
ha superado. Además, se señala que el incumplimiento de este plazo no afecta la validez del
procedimiento, sino que compromete la responsabilidad funcionaria. En relación a la omisión en
la aplicación del Instructivo, se argumenta que este no establece la obligatoriedad de seguir
el orden de tramitación indicado. Por último, se concluye que las solicitudes de diligencias
probatorias solicitadas por la interesada no son necesarias para confirmar o rechazar la
verificación de los hechos que configuran la elusión atribuida en la resolución impugnada. En
resumen, se determina que las controversias planteadas por la reclamante no afectan la
legalidad del procedimiento administrativo.\nEn resumen, el caso trata sobre una reclamación
presentada por un titular de un proyecto contra la Superintendencia del Medio Ambiente (SMA)
en relación con el requerimiento de ingreso al Sistema de Evaluación de Impacto Ambiental
(SEIA). El reclamante alega que la SMA infringió el Instructivo para la Tramitación de
Requerimientos de Ingreso al SEIA al no solicitar previamente el informe del Servicio de
Evaluación Ambiental (SEA) antes de iniciar el procedimiento. Sin embargo, la SMA argumenta
que cumplió con lo establecido en la ley y que el Instructivo no es de carácter obligatorio.
El tribunal concluye que la SMA actuó de acuerdo a la ley y que no se produjo indefensión para
el reclamante. Además, se analiza la omisión de decretar las diligencias probatorias
solicitadas por el interesado y se determina que no causaron indefensión y que eran
innecesarias para el caso. En consecuencia, el tribunal considera que la resolución de ingreso
al SEIA se ajusta a derecho.\nEl Tribunal Ambiental rechazó la reclamación presentada por la
impugnante en contra de la Resolución Exenta N° 427 de la Superintendencia del Medio Ambiente
(SMA). La impugnante solicitaba la realización de diligencias probatorias, pero no especificó
de manera clara la conexión entre dichas diligencias y los hechos que deseaba probar. Además,
no indicó las consecuencias jurídicas que se derivarían de la falta de pronunciamiento de la
autoridad sectorial respecto a los recursos administrativos pendientes de resolución. Por lo
tanto, el Tribunal consideró que la Resolución de la SMA se ajusta a derecho. No se condenó en
costas a la impugnante por considerar que litigó con motivos plausibles.',
            metadata={
                'caratula': 'Asociación Indígena Aymara Salar de Coposa con SMA ',
                'competencia': 'Art. 17 N° 3 de Ley N° 20.600',
                'componente_ambiental': 'Agua',
                'decision_del_tribunal': 'Acoge',
                'descriptores': 'Procedimiento sancionatorio; programa de cumplimiento;
consulta indígena; recurso hídrico; daño ambiental; acuíferos; criterios; integridad;
eficacia; verificabilidad; derecho a los recursos naturales; principio de participación;
motivación; reincidencia ',
                'fechaSentencia': '30-12-2019',
                'nombreProyecto': 'Collahuasi',
                'page': 0,
                'rol': 'R-25-2019',
                'source':
'/home/sjvasconcello/usm-data/ds-senior/cmad-ai-sustentable-01/etl/load/../../data/clean/docs_
with_summary/R-25-2019.txt',
                'tribunal': '1TA',
                'urlSentencia':
'https://www.buscadorambiental.cl/buscador-api/file/view/jurisprudencia/S1TA Rol N°
R-25-2019?id=16'
            }
        ),
        Document(
            page_content='El caso Pascua Lama\n\nRol: "R-5-2018", Rol: "R-5-2018",Rol:
"R-5-2018", fecha 17-09-2020 de Compañía Minera Nevada SpA con Superintendencia del Medio
Ambiente cuyo proyecto corresponde a Pascua Lama. Este caso tiene como principal materia
Sancionatorio ambiental y su coponente ambiental es Agua\n\nThe integration ministers are:
\n1. Cristián Delpiano Lira\n2. Jasna Pavlich Núñez\n3. Fabrizio Queirolo Pellerano\n\nSiendo
sus descriptores: Sancionatorio ambiental, estándar de motivación – proporcionalidad –
principio de non bis in ídem – decaimiento – metodología de determinación de sanciones –
legitimación activa – tasa de descuento – daño ambiental – impacto ambiental – clasificación
de la infracción - circunstancias del art. 40 de la Ley N° 20.417\n\nEl First Environmental
Court Acoge parcialmente la causa en esta ocacion. Los hechos controvertidos son: Ilegalidad
transversal de las sanciones impuestas por la Res. Ex. N°72/2018. De las supuestas
ilegalidades respecto de las sanciones de clausura definitiva aplicadas a cinco de los cargos
formulados. De las supuestas ilegalidades cometidas en la Res. Ex. N°72/2018, respecto de las
sanciones de multa aplicadas, al resto de los cargos formulados. Del supuesto exceso en las
competencias que habría incurrido la SMA respecto de la manera en que pretende que se
implemente la clausura. Respecto a la falta de legitimación activa alegada por la SMA en
contra de Agrícola Dos Hermanos y Agrícola Santa Mónica Limitada. Respecto de la Resolución
Exenta N°70. Respecto de la Res. Ex. N°72. Legitimación activa de los reclamantes de la
Reclamación signada con el rol R-6-2018. Estándar de motivación exigido. Resolución Exenta
N°70, de 2018, de la Superintendencia del Medio Ambiente. Decaimiento del procedimiento
administrativo y la falta de oportunidad en el ejercicio de la potestad sancionatorio. De la
supuesta infracción al principio non bis in ídem. En cuanto a la metodología para la
determinación de sanciones ambientales, la tasa de descuento, y la proporcionalidad de las
sanciones, de la Resolución Exenta N°72/2018.2.7 En cuanto al cálculo de la Tasa de Descuento
descrita en las “Bases Metodológicas para Determinación de Sanciones Ambientales”. El estándar
de proporcionalidad en el ejercicio de la potestad sancionatoria de la Superintendencia del
Medio Ambiente. III. DEL ANÁLISIS DE LAS RECLAMACIONES EFECTUADAS PARA CADA CARGO EN
PARTICULAR ALEGADO RESPECTO DE LA RESOLUCIÓN EXENTA N°72, REALIZADO EN CAUSA R-5-2018. 3.1 De
las supuestas ilegalidades cometidas en la sanción de clausura definitiva aplicada al cargo
23.2. 3.2 De las supuestas ilegalidades cometidas en la sanción de clausura definitiva
aplicada al cargo 23.9. 3.3 De las supuestas ilegalidades cometidas en la sanción de clausura
definitiva aplicada al cargo 23.11. 3.4 De las supuestas ilegalidades cometidas en la sanción
de clausura definitiva aplicada al cargo 4. 3.5 De las supuestas ilegalidades cometidas en la
sanción de clausura definitiva aplicada al cargo N°7. 3.6 De las supuestas ilegalidades
cometidas en la sanción de multa aplicada al cargo 23.8. 3.7 De las supuestas ilegalidades
cometidas en la sanción de multa aplicada al cargo 6. 3.8 De las supuestas ilegalidades
cometidas en la definición de todas las sanciones de multa al establecer un criterio de tasa
de descuento que es muy superior al de las empresas similares. 3.9 Del supuesto exceso en las
competencias que habría incurrido la SMA respecto de la manera en que pretende que se
implemente la clausura. IV. DEL ANÁLISIS DE LAS RECLAMACIONES EFECTUADAS PARA CADA CARGO EN
PARTICULAR ALEGADO RESPECTO DE LA RESOLUCIÓN EXENTA N°72, REALIZADO EN CAUSA R-6-2018. 4.1
Vicio N° 1: Respecto a la omisión de la resolución sancionatoria de ordenar a CMN SpA la
construcción definitiva del Sistema de Manejo de Aguas. 4.2 Vicio N°2: Sobre la falta de
contenido sancionatorio y disuasivo en la aplicación de 5 sanciones de clausura. 4.3 Vicio
N°3: Respecto al descarte ilegal de daño ambiental en las aguas. 4.4 Vicio N°4: De la supuesta
falta de fundamentación de la resolución reclamada en lo que dice relación con la causalidad
en la contaminación de las aguas. 4.5 Vicio N°5. Respecto de la motivación de la SMA para
recalificar y absolver a CMN SpA de los cargos 23.13, 24.2 y 25. 4.6 Vicio N°6: Sobre la
eventual falta de proporcionalidad en la aplicación de las sanciones. 4.7 Vicio N°7: La
resolución recurrida no pondera ni aplica en concreto las circunstancias que contempla el
artículo 40 de la LOSMA para efectos de determinar la sanción a aplicar a cada infracción. 4.8
Del supuesto problema del petitorio de la reclamación.',
            metadata={
                'caratula': 'Compañía Minera Nevada SpA con Superintendencia del Medio
Ambiente',
                'competencia': 'Art. 17 N° 3 de Ley N° 20.600',
                'componente_ambiental': 'Agua',
                'decision_del_tribunal': 'Acoge parcialmente',
                'descriptores': 'Sancionatorio ambiental; estándar de motivación –
proporcionalidad – principio de non bis in ídem – decaimiento – metodología de determinación
de sanciones – legitimación activa – tasa de descuento – daño ambiental – impacto ambiental –
clasificación de la infracción - circunstancias del art. 40 de la Ley N° 20.417',
                'fechaSentencia': '17-09-2020',
                'nombreProyecto': 'Pascua Lama',
                'page': 0,
                'rol': 'R-5-2018',
                'source':
'/home/sjvasconcello/usm-data/ds-senior/cmad-ai-sustentable-01/etl/load/../../data/clean/docs_
with_summary/R-5-2018.txt',
                'tribunal': '1TA',
                'urlSentencia':
'https://www.buscadorambiental.cl/buscador-api/file/view/jurisprudencia/S1TA Rol N°
R-5-2018?id=20'
            }
        )
    ]
}
\end{table}
\end{Verbatim}


