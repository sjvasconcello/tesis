
\chapter{Conclusiones}

\par Tal como dijo Arthur C. Clarke: ``Cualquier tecnología suficientemente avanzada es indistinguible de la magia''. Nos encontramos 
en un período donde la Inteligencia Artificial esta alcanzado capacidades que cada vez nos sorprenden más y nos asustan por igual, donde 
posiblemente nos estamos desilusionando al observar que aún no se logran cosas que imaginamos que a estas alturas ya deberían ser posibles, pero a su vez nos maravillamos viendo cómo se consiguen 
cosas que ni siquiera nos pudimos plantear en un inicio que se lograrían.  

\par El uso comercial de la inteligencia artificial en especial de lo LLM, como fue la cobertura de esta tesis, presenta una cantidad 
muy alta de riesgos, tanto de los que son fáciles de prevenir, como pueden ser los cuantitativos dentro de los que se encuentran prevenir costos excesivos en 
servidores, como los que a su vez son difíciles de prevenir como las alucinaciones.  Es importante para que estas aplicaciones tengan un buen futuro, 
entender el funcionamiento de estas como la manera en que fueron creadas. Los sesgos que presentan estos modelos no dejan de ser un reflejo
de lo que somos como humanidad y que le dimos de alimento a estos modelos para ser entrenados, siendo las respuestas sesgadas entregadas por los modelos un gran reflejo de cómo somos nosotros 
y la manera en la que actuamos, demostrando que al igual que estos modelos somos lo que consumimos.  

\par La industria de la Inteligencia Artificial deja expuesto a absolutamente todos los trabajos desde ahora en adelante en mayor 
o en menor medida, por lo que hay que tener cautela en las decisiones que se toman, pues el riesgo al que nos exponemos es sumamente grande. La 
volatilidad del mercado posiblemente es y será el riesgo más grande por considerar para cualquier tipo de proyecto de este tipo, 
no fue hace mucho que los llamados Prompt engineer serían los profesionales más cotizados en el mercado, incluso mencionados así por el CEO en 
Nvidia \cite{conclusion1}. Sin embargo, estos fueron ya rápidamente reemplazados por los mismos LLM que se supone tenían que domar, debido a la optimización \cite{conclusion2}
o el auto mejoramiento mediante generación de prompt producidos el mismo LLM \cite{conclusion3}, dejando de esa manera obsoleto un rol que hace menos de 
 tres meses a la fecha de publicación de esta tesis sería uno de los roles más importantes en el uso de Inteligencia Artificial Generativa.

\par Finalmente, para cualquier tipo de proyecto sobre o con uso de Inteligencia artificial siempre lo más importante serán los datos y el 
criterio del científico de datos detrás de ellos, porque existirán datos y herramientas, pero sin un conocimiento de mercado al que se apunta, 
realizar cualquier tipo de acción es trabajar en la oscuridad porque sin criterio, trabajar con datos es un trabajo en vano y sin sentido.
