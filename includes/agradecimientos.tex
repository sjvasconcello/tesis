%!TEX root = ../memoria.tex

	Agradezco a quienes contribuyeron para ir mejorando esta plantilla hecha en \LaTeX{}. 
    
    Los aportes y comentarios de distintas personas en el \href{http://www.industrias.usm.cl}{Departamento de Industrias} fueron muy útiles para que este documento puede ser ocupado para mejorar la presentación de tesis y memorias del Departamento (y la universidad).
    
\vspace{20mm}
\begin{framed}
\noindent\textbf{\color{red}Para el impaciente ...}

Por favor ocupar \inlinecode{git}:

\inlinecode{git clone https://github.com/jaimercz/utfsm-thesis.git}

Para los interesados en \inlinecode{git} revisar \parencite{git2017}.

Abra el archivo de configuración \inlinecode{config.tex} para cambiar título, autor, fecha, etc. de la portada y del documento en general.

Abra  y compile el documento maestro \inlinecode{memoria.tex}.

\begin{Verbatim}[fontsize=\small]
        $ pdflatex memoria.tex
        $ biber memoria
        $ pdflatex memoria.tex
        $ pdflatex memoria.tex
\end{Verbatim}

Esta version ocupa \inlinecode{biber} en lugar de \inlinecode{natbib / bibtex}:

Si hay errores, verifique primero que todos los paquetes \LaTeX{} han sido instalados.

Si desea omitir alguna sección (dedicatoria, agradecimientos, etc.), revise el documento maestro \inlinecode{memoria.tex} y agregue o comente (o elimine) las líneas correspondientes.

Por ejemplo, para eliminar esta sección, borre las líneas:

\begin{Verbatim}[frame=lines, label=\inlinecode{memoria.tex} (extracto)
, fontsize=\footnotesize
, baselinestretch=1
, formatcom=\color{gray}]
\section*{Agradecimientos}
\insertFile[plain]{agradecimientos}}
\end{Verbatim}
\end{framed}
