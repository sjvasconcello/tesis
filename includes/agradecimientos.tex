%!TEX root = ../memoria.tex
Aunque resulta casi imposible el agradecer a todos los que han estado conmigo en esta epopeya que ha sido salir de la universidad, es necesario hacer el intento, porque no ser agradecido tanto del apoyo que he recibido, además de ser inconsciente, sería extremadamente presumido. Todos somos el promedio de las personas que frecuentamos y yo definitivamente he tenido mucha suerte.

En ámbito laboral, muchísimas gracias Varder Johnson, en Chaski e Ignacio Quintana, en Moneda Asset Managment, quienes confiaron en mí y me permitieron ser sus colegas.  También a Esteban Róman, Claudio Orrego y Lucas Vallejo, que fueron mis compañeros de trabajo, de los cuales aprendí mucho más de lo que alguna vez imaginé.

Y aunque pueda resultar atípico, también agradezco a las comunidades online que formo parte, especialmente a la comunidad de Platzi y de en\_coders, que además de ayudar cuando se necesita, también brindan compañía en el solitario camino del estudio y aprendizaje autodidacta.  

En mi vida universitaria, le agradezco a los profesores: Fernando Díaz, Maria Pía Santibáñez, Francisca González, Rodrigo Ortega, Joaquin Dagnino y Norman Dabner por permitirme ser su ayudante y aprender mucho de ellos. Y a Thierry de Saint-Pierre, por confiar de manera tan ciega en dos estudiantes y permitirnos trabajar en proyectos que - hasta ese momento - no sabía que podían realizarse. 

Por otro lado, no puedo dejar de mencionar a quienes alimentaron mi curiosidad y me alentaron a estudiar. Agradezco a Jorge Bayer, por enseñarme a programar y cambiarme la vida desde el primer script. A Mario Vega, el mejor profesor de matemáticas que existe y que además de ayudarme a confiar en mí, me enseñó lo hermosas que pueden llegar a ser las matemáticas.

Gracias a mis amigos del CMAD: Juan, Benjamín, Constanza, María Olga, Bastián, Anais y Jorge.  Gracias por compartir conmigo lo nerd que puedo llegar a ser. A mis amigas: Paulina, Gabriela, Isidora, Ximena, Valeria y Yanira, gracias por estar ahí para mí, por ser mis compañeras de trabajo y mi sostén emocional en la universidad. 

Un especial agradecimiento, a Ignacio Cortes, Marco Figueroa y Marcelo Escudero, gracias por ser mis mejores amigos de toda la vida, los quiero mucho y no los cambiaría por nada. 

A mis abuelos, siempre atentos de cómo iba y en especial a ti mamita, vuelvo a agradecer y pido perdón por no alcanzar a titularme antes de que partieras, espero que si existe otro plano puedas ver que lo logre. También a Rayén y mi sobrina Tamara, atentas y presentes. Gracias también a mi tía Luchita, tío Marco, Eli, Benjamín y José Pablo, porque a veces la familia no solo es de sangre.


Para terminar, siempre estará mi familia. Mi madre que ha estado en las buenas y en las malas, siempre en eterna vigilia para apoyarme cuando lo necesitaba, no sería nada si ti mamá. A mi padre,  siempre presente, siempre agradeceré la paciencia y la oportunidad que me diste de equivocarme para ser lo que soy ahora.

Mi hermana, mi apoyo en todo momento, confidente y la base emocional en mis peores momentos, sin ti no hubiese ni siquiera podido entrar en la universidad. Son lo más importante en mi vida, siempre faltarán palabras para agradecerles. Y finalmente, para terminar un ciclo más, quiero decir que esto también va por ti Cristobal, porque, aunque el destino nos separó, siempre vivirás en mí. Te dedico este trabajo, porque para mí hoy nos titulamos los dos.

Muchísimas gracias a todos los que son parte de mi vida, son mi improbabilidad matemática favorita.