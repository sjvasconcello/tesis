
\vspace{10mm}

Este trabajo investiga los riesgos inherentes a la creación y uso de aplicaciones basadas en Modelos Grandes de Lenguaje (LLM) en la industria. Su foco es en Procesamiento del Lenguaje Natural (NLP) específicamente la generación de texto, excluyendo otras formas de inteligencia artificial generativa. El proyecto se basa en la experiencia de desarrollar una aplicación utilizando LLM y analiza los riesgos que pueden afectar tanto al equipo de desarrollo como a los resultados obtenidos. Se determinan los factores de riesgo en la creación y uso de estas aplicaciones, utilizando como caso de estudio un proyecto de búsqueda de jurisprudencia en tribunales ambientales. La metodología empleada incluye la creación del proyecto, un ejemplo práctico de uso y la evaluación de riesgos en cada etapa del proceso. El objetivo es proporcionar una estructura para aplicaciones que usen LLM, analizar los problemas y riesgos asociados con el uso de información para alimentar estos modelos, incluyendo un proceso completo de ETL (Extract, Transform, Load).

\paragraph{Palabras Clave.}
Modelos Grandes de Lenguaje (LLM),
Generación de Texto,
Riesgo,
Inteligencia Artificial,
Proceso de ETL,
Jurisprudencia Ambiental

\vspace*{1cm}
% \subsection*{¡Importante! [LEAME]}
%
%\subsubsection*{Impresión por un solo lado.}
%A partir del año 2016, el Departamento de Ingenieria Comercial sólo requiere la entrega digital de los archivos de memorias y tesis. Por este motivo, este documento está preparado para ser impreso por un solo lado de una hoja (\emph{``oneside''}), y facilitar así su lectura en pantallas. Esta configuración es parte de archivo de clase \inlinecode{thesis_utfsm.cls}.

%\subsubsection*{Codificación de caracteres.}
%Todos los archivos \inlinecode{*.tex} de esta plantilla han sido preparados ocupando la codificación de caracteres por defecto \emph{unicode} (UTF-8). Windows (y algunas versiones de OSX) ocupan otro tipo de codificación (ej. \emph{Windows-1252} o \emph{Mac Roman}).

%Si deseas ocupar esta plantilla y encuentras problemas con los caracteres acentuados, entonces puedes optar por una de estas tres alternativas:
%\begin{enumerate}[i)]
%    \item cambiar tu editor (TexMaker, TexStudio, TexShop, etc.) para que ocupe UTF-8 como codificación de caracteres por defecto; o
%    
%    \item cambiar la codificación de cada documento \inlinecode{*.tex} para que ocupe la codificación nativa de tu sistema operativo; y, modificar el archivo \inlinecode{config.tex} la línea que dice:
%    
%    OSX, Linux: \inlinecode{\\usepackage[utf8x]\{inputenc\}}

%    Windows:    \inlinecode{\\usepackage[latin1]\{inputenc\}}
    
%    Overleaf:   \inlinecode{\\usepackage[utf8]\{inputenc\}} \url{https://overleaf.com}
    
%    \item escribir todo ocupando caracteres pre-acentuados (ej. \inlinecode{\\'a} en lugar de á).
%\end{enumerate}
