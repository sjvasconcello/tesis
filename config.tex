%!TEX root = memoria.tex

%---------------------------------------------------------------------------
%%% CONFIGURACIÓN
%---------------------------------------------------------------------------
%
%%% CODIFICACIÓN DE CARACTERES
% Este documento está escrito usando caracteres Unicode (UTF8)
% Por lo que la siguiente línea es necesaria para reconocer los acentos
% y otros caracteres en español.
% Si ve caracteres extraños en el PDF (en Windows o MAC) pruebe 
% con alguna de estas líneas:
\usepackage[utf8]{inputenc}     % Overleaf
%\usepackage[utf8x]{inputenc}   % *nix / Linux / MacOSX
%\usepackage[latin1]{inputenc}  % Windows (MacOSX)


\newcommand{\TheTitle}{%
    TÍTULO DE MEMORIA (EL TÍTULO SÓLO PUEDE TENER UN MÁXIMO DE 3 LÍNEAS)
}%
\newcommand{\TheAuthor}          {SANTIAGO JESÚS VASCONCELLO ACUÑA}
\newcommand{\TheGrade}           {INGENIERO COMERCIAL} % Elegir "O" ó "A"
\newcommand{\TheCity}            {SANTIAGO}
\newcommand{\TheDate}            {Diciembre 2023}
\newcommand{\TheAdvisor}         {SR. PABLO ISLA}
\newcommand{\TheCoAdvisor}       {SR. THIERRY DE SAINT PIERRE.} % Thierry A De Saint Pierre
%\newcommand{\TheScndCoAdvisor}   {SRTA. XXXXXXX XXXXXXXX X.} % Opcional

% Marca de agua, puede ser deshabilitada para impresión rápida
\insertWatermark{figures/logousm_watermark.jpg}
%---------------------------------------------------------------------------


%---------------------------------------------------------------------------
%%% No editar (¡Ver licencia!) (MIT License, 2016)
%---------------------------------------------------------------------------
\hypersetup{  
    pdfinfo={  
        Subject={Tesis Departamento de Ingeniería Comercial, UTFSM},
        Keywords={Tesis} {Departamento de Ingeniería Comercial} {UTFSM},
        Producer={JCR LaTeX Templates, http://www.rubin-de-celis.com/},
        Licence={http://www.rubin-de-celis.com/LICENSE},
        pdfpagemode=FullScreen,
        pdfmenubar=false,
        pdftoolbar=false
    }  
}
\hypersetup{  
    pdfinfo={  
        Title={\TheTitle},
        Author={\TheAuthor}
    }  
}
%---------------------------------------------------------------------------
